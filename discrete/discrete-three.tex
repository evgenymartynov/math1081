\documentclass[11pt]{article}

\usepackage{fullpage}
\usepackage{amsmath,amsfonts,amssymb,amsthm}

\newtheorem*{theorem}{Theorem}

\newcommand{\R}{\mathbb{R}}
\newcommand{\Q}{\mathbb{Q}}
\newcommand{\Z}{\mathbb{Z}}

% List styles
\renewcommand{\theenumii}{\roman{enumii}}
\renewcommand{\labelenumii}{(\theenumii)}

\begin{document}
    \begin{center}
        \Large \bf MATH1081 Test 2 past solutions
    \end{center}

    \begin{center}
        \large \bf \copyright UNSW MATHSOC 2012
    \end{center}

    \vspace{10mm}

    These solutions were written by Johann Blanco and typed up by Evgeny
    Martynov. Please be ethical with this resource. It is for the use of
    MATHSOC members, so {\bf do not repost it on other forums or groups
    without asking for permission}. If you appreciate this resource,
    please consider supporting us by coming to our events and buying our
    T-shirts! Also, happy studying :)

    We cannot guarantee that our working is correct, or that it would
    obtain full marks - please notify us of any errors or typos at
    unswmathsoc@gmail.com, or on our Facebook page. There are sometimes
    multiple methods of solving the same question. Remember that in the
    real class test, you will be expected to explain your steps and
    working out.

    This document was written in mathematical typesetting language
    called \LaTeX(pronounced lay-teck). You can learnt a lot about how
    \LaTeX works by downloading the code for existing documents and
    fiddling around with it. If you would like to try this, send an
    email to unswmathsoc@gmail.com and  we can send you the code for
    this file.

    \newpage
    \centering{\bf \Large 2008 Version 3a}

\begin{enumerate}
% Question 1, truth tables
\item
    \begin{enumerate}
    \item 
        $\sim p \rightarrow (q \wedge p)$

        \begin{tabular}{c|c|c|c|c}
        $p$ & $q$ & $\sim p$ & $q \wedge p$ & $\sim p \rightarrow (q \wedge p)$ \\
        \hline
        T & T & F & T & T \\
        T & F & F & F & T \\
        F & T & T & F & F \\
        F & F & T & F & F \\
        \end{tabular}

        \vspace{3mm}

        $(p \wedge q) \rightarrow q$

        \begin{tabular}{c|c|c|c}
        $p$ & $q$ & $p \wedge q$ & $(p \wedge q) \rightarrow q$ \\
        \hline
        T & T & T & T \\
        T & F & F & T \\
        F & T & F & T \\
        F & F & F & T \\
        \end{tabular}

    \item
        First off, note that the second formula cannot imply the first
        when $p = q = F$.

        On the other hand, the first formula \textbf{does} imply the
        second, since when the first is true, the second is true also.

        This can be better explained using this truth table:

        \begin{tabular}{c|c|c|c|c}
        $p$ & $q$ & $\sim p \rightarrow (q \wedge p)$ &
            $(p \wedge q) \rightarrow q$ & first $\rightarrow$ second \\
        \hline
        T & T & T & T & T \\
        T & F & T & T & T \\
        F & T & F & T & T \\
        F & F & F & T & T \\
        \end{tabular}

        Hence, the first formula logically implies the second.

    \end{enumerate}

% Question 2, log_6 (11) not in Q
\item
    \begin{theorem}
        $\log_6 11$ is irrational.
    \end{theorem}
    \begin{proof}
        We prove by contradiction.

        Suppose that $\log_6 11$ is rational, that is, there exist
        integers $p$ and $q$ such that $log_6 11 = \frac{p}{q}$.
        Without loss of generality, we can assume $p, q > 0$.

        Then,
        \begin{align*}
            \log_6 11 &= \frac{p}{q} \\
            11 &= 6^\frac{p}{q} \\
            11^q &= 6^p \\
        \end{align*}

        which is a contradiction, as the LHS is always odd and RHS is
        always even.

        Hence, our assumption was false, and it follows that $\log_6 11$
        is irrational, thus completing the proof.
    \end{proof}

% Question 3, the one with prime numbers.
\item
    \begin{theorem}
        $q(n) = 11n^2 + 32n$ is a prime number for two integer values
        of $n$, and composite for every other integer values of $n$.
    \end{theorem}
    \begin{proof}
        If $q(n)$ is prime, then it has exactly two factors, which are
        1 and $q(n)$.

        Factorising $q(n)$, we obtain $q(n) = n(11n+32)$.
        From the definition, it is necessary that at least one of
        $n = \pm 1$ or $11n + 32 = \pm 1$ are true.

        We now consider the possible cases.

        If $n = 1$,  then $q(n) = 43$, which is prime.

        If $n = -1$, then $q(n) = -21$, which is not prime.

        If $11n + 32 = 1$, then $n = \frac{-31}{11} \not \in \Z$.
        This contradicts the constraint of $n \in \Z$ and hence is not
        possible.

        If $11n + 32 = -1$, then $n = -3$, and hence $q(n) = 3$,
        which is prime.

        By exhaustion of possible cases where $q(n)$ can be prime, we
        conclude that $q(n)$ is prime in only two cases (namely, $n=1$
        or $n=-3$), and composite for every other integer value of $n$.
    \end{proof}
\end{enumerate}

    \newpage
    \centering{\bf \Large 2008 Version 1a}

\begin{enumerate}
% Question 1, truth tables
\item
    \begin{enumerate}
    \item 
        $(p \rightarrow (\sim q)) \wedge r$

        \begin{tabular}{c|c|c|c|c}
            $p$ & $q$ & $r$ & $p \rightarrow (\sim q)$ &
                $(p \rightarrow (\sim q)) \wedge r$ \\
            \hline
            T & T & T & F & F \\
            T & T & F & F & F \\
            T & F & T & T & T \\
            T & F & F & T & F \\
            F & T & T & T & T \\
            F & T & F & T & F \\
            F & F & T & T & T\ \\
            F & F & F & T & F \\
        \end{tabular}

        \vspace{3mm}

        $q \rightarrow ((\sim p) \wedge r)$

        \begin{tabular}{c|c|c|c|c}
            $p$ & $q$ & $r$ & $(\sim p) \wedge r$ &
                $q \rightarrow ((\sim p) \wedge r)$ \\
            \hline
            T & T & T & F & F \\
            T & T & F & F & F \\
            T & F & T & F & T \\
            T & F & F & F & T \\
            F & T & T & T & T \\
            F & T & F & F & F \\
            F & F & T & T & T \\
            F & F & F & F & T \\
        \end{tabular}

    \item
        First off, note that the second cannot imply the first when
        $p = q = r = F$.

        On the other hand, the first \textbf{does} imply the second,
        which can be seen from this truth table:

        \begin{tabular}{c|c|c|c|c|c}
            $p$ & $q$ & $r$ & $(p \rightarrow (\sim q)) \wedge r$ &
                $q \rightarrow ((\sim p) \wedge r)$ &
                first $\rightarrow$ second \\
            \hline
            T & T & T & F & F & T \\
            T & T & F & F & F & T \\
            T & F & T & T & T & T \\
            T & F & F & F & T & T \\
            F & T & T & T & T & T \\
            F & T & F & F & F & T \\
            F & F & T & T & T & T \\
            F & F & F & F & T & T \\
        \end{tabular}

        Hence, the first formula logically implies the second.
    \end{enumerate}

% Question 2: m!n! < (m+n)!
\item
    \begin{theorem}
        If $m$ and $n$ are positive integers, then $m!n! < (m+n)!$
    \end{theorem}
    \begin{proof}
        Let $m, n \in \Z^+$.

        Now,
        \begin{align*}
            m!n! &= (1 \times 2 \times 3 \times \dots \times m)
                    (1 \times 2 \times 3 \times \dots \times n) \\
            &< 1 \times 2 \times 3 \times \dots \times m
                    (m+1)(m+2)(m+3)\dots(m+n) \\
            &= (m+n)!
        \end{align*}
        as $1 < m+1, 2 < m+2, \dots, n < m+n$.

        It follows that $m!n! < (m+n)!$, thus completing the proof.
    \end{proof}

% Question 3: ax cos pi x
\item
    \begin{theorem}
        If $a \in \R^+$, then the equation $ax = \cos \pi x$ has exactly
        one solution $x$ such that $0 \leq x \leq 1$.
    \end{theorem}
    \begin{proof}
        Let $a \in R^+$. Let $f(x) = ax - \cos \pi x$.

        Now, $f(0) = -1$ and $f(1) = a + 1$.

        Clearly, $f$ is continuous on the interval $[0, 1]$.
        Thus, as $-1 \leq 0 \leq a+1$, by the intermediate value
        theorem, there exists some $c \in [0, 1]$ such that $f(c) = 0$.

        That is, there is \textit{a} solution for $ax = \cos \pi x$ on
        the specified interval.

        Further, on the interval $[0, 1]$,
        $$f'(x) = a + \pi \sin \pi x > 0$$
        that is, $f$ is increasing, and so it can have at most one root
        in that region.

        It follows that $ax = \cos \pi x$ has exactly one solution on
        the interval $[0, 1]$, thus completing the proof.
    \end{proof}
\end{enumerate}

    \newpage
    \centering{\bf \Large 2009 Version 2a}

\begin{enumerate}
% Question 1, logical equivalences
\item ~

    \begin{tabular}{ll}
        $p \rightarrow (\sim (q \wedge (\sim p)))$ & \\
        $\Leftrightarrow(\sim p) \vee (\sim (q \wedge (\sim p)))$ &
            as $v \rightarrow u \Leftrightarrow (\sim v) \vee u$ \\
        $\Leftrightarrow\sim(p \wedge (q \wedge (\sim p)))$ &
            De Morgan's law \\
        $\Leftrightarrow\sim(p \wedge (\sim p) \wedge q)$ &
            Commutative and associative laws \\
        $\Leftrightarrow\sim(\mathbb{F} \wedge q)$ & Law of negation \\
        $\Leftrightarrow\sim(\mathbb{F})$ & Domination law \\
        $\Leftrightarrow\mathbb{T}$ & Negation of a contradiction \\
    \end{tabular}

% Question 2: induction
\item
    \begin{theorem}
        If $n \in \Z^+$, then
        $$(1 \times 2) + (2 \times 5) + \dots + n(3n-1) = n^2(n+1)$$
    \end{theorem}
    \begin{proof}
        We prove by induction.

        When $n = 1$, LHS $= 1 \times 2 = 2$, RHS $= 1^2 \times 2 = 2$,
        and so the theorem is true for $n=1$.

        Assume the theorem for $n=k$, where $k \in \Z^+$. That is:
        $$(1 \times 2) + (2 \times 5) + \dots + k(3k-1) = k^2(k+1)$$

        Consider $n = k+1$. We have:
        \begin{align*}
            \textrm{LHS} &= (1 \times 2) + \dots + k(3k-1) + (k+1)(3k+2) \\
            &= k^2(k+1) + (k+1)(3k+2) & \textrm{(by assumption)} \\
            &= (k+1)(k^2 + 3k + 2) \\
            &= (k+1)(k+1)(k+2) & \textrm{(factorising)} \\
            &= (k+1)^2((k+1)+1) \\
            &= \textrm{RHS}
        \end{align*}

        Hence, if the theorem is true for $n=k$ for some $k \in \Z^+$,
        then it is also true for $n=k+1$.

        Therefore, the theorem is proved for all positive integers $n$,
        by induction, and the proof is complete.
    \end{proof}

% Question 3: irrationality
\item
    \begin{theorem}
        If $x \in \R$ and $2x^2 - 3 = 0$, then $x$ is irrational.
    \end{theorem}
    \begin{proof}
        We prove by contradiction.

        Let $x \in \R$ and $2x^2 - 3 = 0$.
        Solving, we obtain
        $$x^2 = \frac{3}{2}$$

        Suppose that $x$ is rational, that is, it can be written as an
        irreducible ratio of two integers, $x = \frac{p}{q}$ for some
        $p, q \in \Z \backslash \{0\}$.

        Then, $x^2 = \frac{p^2}{q^2} = \frac{3}{2}$ and thus
        $$2p^2 = 3q^2$$

        Clearly, the LHS is even, and so $q^2$ must be even also.
        This implies that $q$ is even, and hence $q = 2r$ for some
        $r \in \Z$.

        Simplifying,
        $$p^2 = 2 \times 3 \times r^2$$

        By similar argument, we see that $p$ must be even, and hence
        $p$ and $q$ share a common factor of 2.
        But since we assumed that $p$ and $q$ form an irreducible
        fraction, this is a contradiction.

        Thus, the initial assumption was wrong.

        It follows that $x$ is irrational, and the proof is complete.
    \end{proof}
\end{enumerate}

    \newpage
    \centering{\bf \Large 2009 Version 1a}

\begin{enumerate}
% Question 1, logic
\item
    \begin{enumerate}
    \item Let
        \begin{align*}
            m &= \textrm{``I earn some money''} \\
            h &= \textrm{``I go for a holiday this summer''} \\
            w &= \textrm{``I work this summer''} \\
        \end{align*}
        The argument is:

        \begin{tabular}{c}
            $m \rightarrow h$ \\
            $h \vee w$ \\
            \hline
            $\therefore (\sim h) \rightarrow (\sim m) \wedge w$ \\
        \end{tabular}

    \item
        Using a truth table, we consider critical rows where the
        hypotheses are true:

        \begin{tabular}{c|c|c|c|c|c|c}
        $m$ & $h$ & $w$ & $m \rightarrow h$ & $h \vee w$ &
            $(\sim m) \wedge w$ &
            $(\sim h) \rightarrow (\sim m) \wedge w$ \\
        \hline
        T & T & T & T & T & F & \textbf{T} \\
        T & T & F & T & T & F & \textbf{T} \\
        T & F & T & F & * & * & * \\
        T & F & F & F & * & * & * \\
        F & T & T & T & T & T & \textbf{T} \\
        F & T & F & T & T & F & \textbf{T} \\
        F & F & T & T & T & T & \textbf{T} \\
        F & F & F & T & F & * & * \\
        \end{tabular}

        The conclusion is always true whenever the hypotheses are true,
        and therefore the above argument is logically valid.
    \end{enumerate}

% Question 2: rationals
\item
    \begin{theorem}
        Between any two different rational numbers there is another
        rational number.
    \end{theorem}
    \begin{proof}
        Let $x, y$ be different rational numbers.
        Without loss of generality, assume $x < y$.

        Then we have $\frac{x+x}{2} < \frac{x+y}{2} < \frac{y+y}{2}$.

        Since $x, y$ are rational, and rational numbers are closed under
        addition and division, it follows that $\frac{x+y}{2}$ is also a
        rational number; and the proof is finished.
    \end{proof}

% Question 3: divisibility
\item
    \begin{theorem}
        If $n$ is a positive integer, then $4^{2n} + 10n - 1$ is a
        multiple of 25.
    \end{theorem}
    \begin{proof}
        We prove by induction.

        When $n = 1$, $4^{2n} + 10n - 1 = 25$, and so the theorem holds for
        $n = 1$.

        Assume the theorem holds for $n = k$, where $k \in \Z^+$.
        That is,
        $$4^{2k} + 10k - 1 = 25A$$
        for some $A \in \Z^+$.

        Now consider $n = k+1$.
        We have
        \begin{align*}
            4^{2k+2} + 10(k+1) - 1 &= 16 \times 4^{2k} + 10k + 9 \\
            &= 16 \times (4^{2k} + 10k - 1) - 150k + 25 \\
            &= 16 \times 25A + 25(1 - 6k) & \textrm{(by assumption)} \\
            &= 25(1 - 6k + 16A)
        \end{align*}
        which is divisible by 25, since $k, A$ are integers.

        Hence, if theorem holds for $n=k, k \in \Z^+$, it also holds for
        $n = k+1$.

        Therefore the theorem is true by induction, completing the proof.
    \end{proof}
\end{enumerate}

    \newpage
    \centering{\bf \Large 2009 Version 1a}

\begin{enumerate}
% Question 1, logic
\item ~

    \begin{tabular}{l l}
    $(p \rightarrow q) \wedge (q \rightarrow (\sim p \vee r))$ &  \\
    $\Leftrightarrow ((\sim p) \vee q) \wedge ((\sim q) \vee (\sim p \vee r))$
        & as $v \rightarrow u \Leftrightarrow (\sim v) \vee u$ \\
    $\Leftrightarrow ((\sim p) \vee q) \wedge ((\sim p) \vee (\sim q \vee r))$
        & Associative, commutative laws \\
    $\Leftrightarrow (\sim p) \vee (q\, \wedge (\sim q \vee r))$
        & Distributive law \\
    $\Leftrightarrow (\sim p) \vee ((q\, \wedge \sim q) \vee (q \wedge r))$
        & Distributive law \\
    $\Leftrightarrow (\sim p) \vee (\mathbb{F} \vee (q \wedge r))$
        & Law of negation \\
    $\Leftrightarrow (\sim p) \vee (q \wedge r)$
        & Identity law \\
    $\Leftrightarrow p \rightarrow (q \wedge r)$
        & as $v \rightarrow u \Leftrightarrow (\sim v) \vee u$ \\
    \end{tabular}

    as required.

% Question 2: equality proof
\item
    \begin{theorem}
        If $n \in \Z^+$, tehn
        $$(n+1)(n+2)\dots(2n) = 2^n \times 1 \times 3 \times \dots \times (2n-1)$$
    \end{theorem}
    \begin{proof}
        Let $n \in \Z^+$.

        Now,
        \begin{align*}
            \textrm{LHS} &= (n+1)(n+2)\dots(2n) \\
            &= \frac{1 \times 2 \times 3 \times \dots \times n \times (n+1) \times \dots \times (2n)}{1 \times 2 \times 3 \times \dots \times n} \\
            &= \frac{(2n)!}{n!}
        \end{align*}

        And,
        \begin{align*}
            \textrm{RHS} &= 2^n \times 1 \times 3 \times \dots \times (2n-1) \\
            &= 2^n \times \left( \frac{1 \times 2 \times 3 \times \dots \times 2n}{2 \times 4 \times 6 \times \dots \times 2n} \right) \\
            % kill me now
            &= (2n)! \times\frac
            {\overbrace{2 \times 2 \times \dots \times 2}^{n \textrm{ terms}}}
            {\underbrace{2 \times 4 \times 6 \times \dots \times 2n}_{n \textrm{ terms}}} \\
            &= (2n)! \times \frac{1}{1 \times 2 \times 3 \times \dots \times n} \\
            &= \frac{(2n)!}{n!} \\
            &= \textrm{LHS}
        \end{align*}

        The result follows, thus completing the proof.
    \end{proof}

% Question 3: irrationality
\item
    \begin{theorem}
        If $n \in \Z^+$, then $\sqrt{4n-2}$ is irrational.
    \end{theorem}
    \begin{proof}
        We prove this by contradiction.

        Let $n \in \Z^+$.
        Suppose that $\sqrt{4n-2}$ is rational, that is, it can be
        written as $\frac{p}{q}$, with $p, q \in \Z$.

        Without loss of generality, we assume that $p$ and $q$ share no
        common factors.

        Then,
        \begin{align*}
            \sqrt{4n-2} &= \frac{p}{q} \\
            4n-2 &= \frac{p^2}{q^2} \\
            2(2n-1) &= \frac{p^2}{q^2} \\
            \Rightarrow 2(2n-1)q^2 &= p^2
        \end{align*}

        (Since the above are integers) It follows that $p^2$ is even,
        and so $p$ is also even.
        Suppose $p = 2m$, for some $m \in \Z$.

        Then we have,
        \begin{align*}
            2(2n-1)q^2 &= p^2 \\
            2(2n-1)q^2 &= 4m^2 \\
            (2n-1)q^2 &= 2m^2 \\
        \end{align*}

        Similarly, we see that $q$ is also even.
        But since $p$ and $q$ share no common factors, this is a
        contradiction.

        Thus, the initial assumption was wrong, and it follows that
        $\sqrt{4n-2}$ is irrational for all $n \in \Z^+$, completing
        the proof.
    \end{proof}
\end{enumerate}

\end{document}
