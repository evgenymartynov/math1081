\centering{\bf \Large 2009 Version 1a}

\begin{enumerate}
% Question 1, logic
\item
    \begin{enumerate}
    \item Let
        \begin{align*}
            m &= \textrm{``I earn some money''} \\
            h &= \textrm{``I go for a holiday this summer''} \\
            w &= \textrm{``I work this summer''} \\
        \end{align*}
        The argument is:

        \begin{tabular}{c}
            $m \rightarrow h$ \\
            $h \vee w$ \\
            \hline
            $\therefore (\sim h) \rightarrow (\sim m) \wedge w$ \\
        \end{tabular}

    \item
        Using a truth table, we consider critical rows where the
        hypotheses are true:

        \begin{tabular}{c|c|c|c|c|c|c}
        $m$ & $h$ & $w$ & $m \rightarrow h$ & $h \vee w$ &
            $(\sim m) \wedge w$ &
            $(\sim h) \rightarrow (\sim m) \wedge w$ \\
        \hline
        T & T & T & T & T & F & \textbf{T} \\
        T & T & F & T & T & F & \textbf{T} \\
        T & F & T & F & * & * & * \\
        T & F & F & F & * & * & * \\
        F & T & T & T & T & T & \textbf{T} \\
        F & T & F & T & T & F & \textbf{T} \\
        F & F & T & T & T & T & \textbf{T} \\
        F & F & F & T & F & * & * \\
        \end{tabular}

        The conclusion is always true whenever the hypotheses are true,
        and therefore the above argument is logically valid.
    \end{enumerate}

% Question 2: rationals
\item
    \begin{theorem}
        Between any two different rational numbers there is another
        rational number.
    \end{theorem}
    \begin{proof}
        Let $x, y$ be different rational numbers.
        Without loss of generality, assume $x < y$.

        Then we have $\frac{x+x}{2} < \frac{x+y}{2} < \frac{y+y}{2}$.

        Since $x, y$ are rational, and rational numbers are closed under
        addition and division, it follows that $\frac{x+y}{2}$ is also a
        rational number; and the proof is finished.
    \end{proof}

% Question 3: divisibility
\item
    \begin{theorem}
        If $n$ is a positive integer, then $4^{2n} + 10n - 1$ is a
        multiple of 25.
    \end{theorem}
    \begin{proof}
        We prove by induction.

        When $n = 1$, $4^{2n} + 10n - 1 = 25$, and so the theorem holds for
        $n = 1$.

        Assume the theorem holds for $n = k$, where $k \in \Z^+$.
        That is,
        $$4^{2k} + 10k - 1 = 25A$$
        for some $A \in \Z^+$.

        Now consider $n = k+1$.
        We have
        \begin{align*}
            4^{2k+2} + 10(k+1) - 1 &= 16 \times 4^{2k} + 10k + 9 \\
            &= 16 \times (4^{2k} + 10k - 1) - 150k + 25 \\
            &= 16 \times 25A + 25(1 - 6k) & \textrm{(by assumption)} \\
            &= 25(1 - 6k + 16A)
        \end{align*}
        which is divisible by 25, since $k, A$ are integers.

        Hence, if theorem holds for $n=k, k \in \Z^+$, it also holds for
        $n = k+1$.

        Therefore the theorem is true by induction, completing the proof.
    \end{proof}
\end{enumerate}
