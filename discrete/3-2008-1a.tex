\centering{\bf \Large 2008 Version 1a}

\begin{enumerate}
% Question 1, truth tables
\item
    \begin{enumerate}
    \item 
        $(p \rightarrow (\sim q)) \wedge r$

        \begin{tabular}{c|c|c|c|c}
            $p$ & $q$ & $r$ & $p \rightarrow (\sim q)$ &
                $(p \rightarrow (\sim q)) \wedge r$ \\
            \hline
            T & T & T & F & F \\
            T & T & F & F & F \\
            T & F & T & T & T \\
            T & F & F & T & F \\
            F & T & T & T & T \\
            F & T & F & T & F \\
            F & F & T & T & T\ \\
            F & F & F & T & F \\
        \end{tabular}

        \vspace{3mm}

        $q \rightarrow ((\sim p) \wedge r)$

        \begin{tabular}{c|c|c|c|c}
            $p$ & $q$ & $r$ & $(\sim p) \wedge r$ &
                $q \rightarrow ((\sim p) \wedge r)$ \\
            \hline
            T & T & T & F & F \\
            T & T & F & F & F \\
            T & F & T & F & T \\
            T & F & F & F & T \\
            F & T & T & T & T \\
            F & T & F & F & F \\
            F & F & T & T & T \\
            F & F & F & F & T \\
        \end{tabular}

    \item
        First off, note that the second cannot imply the first when
        $p = q = r = F$.

        On the other hand, the first \textbf{does} imply the second,
        which can be seen from this truth table:

        \begin{tabular}{c|c|c|c|c|c}
            $p$ & $q$ & $r$ & $(p \rightarrow (\sim q)) \wedge r$ &
                $q \rightarrow ((\sim p) \wedge r)$ &
                first $\rightarrow$ second \\
            \hline
            T & T & T & F & F & T \\
            T & T & F & F & F & T \\
            T & F & T & T & T & T \\
            T & F & F & F & T & T \\
            F & T & T & T & T & T \\
            F & T & F & F & F & T \\
            F & F & T & T & T & T \\
            F & F & F & F & T & T \\
        \end{tabular}

        Hence, the first formula logically implies the second.
    \end{enumerate}

% Question 2: m!n! < (m+n)!
\item
    \begin{theorem}
        If $m$ and $n$ are positive integers, then $m!n! < (m+n)!$
    \end{theorem}
    \begin{proof}
        Let $m, n \in \Z^+$.

        Now,
        \begin{align*}
            m!n! &= (1 \times 2 \times 3 \times \dots \times m)
                    (1 \times 2 \times 3 \times \dots \times n) \\
            &< 1 \times 2 \times 3 \times \dots \times m
                    (m+1)(m+2)(m+3)\dots(m+n) \\
            &= (m+n)!
        \end{align*}
        as $1 < m+1, 2 < m+2, \dots, n < m+n$.

        It follows that $m!n! < (m+n)!$, thus completing the proof.
    \end{proof}

% Question 3: ax cos pi x
\item
    \begin{theorem}
        If $a \in \R^+$, then the equation $ax = \cos \pi x$ has exactly
        one solution $x$ such that $0 \leq x \leq 1$.
    \end{theorem}
    \begin{proof}
        Let $a \in R^+$. Let $f(x) = ax - \cos \pi x$.

        Now, $f(0) = -1$ and $f(1) = a + 1$.

        Clearly, $f$ is continuous on the interval $[0, 1]$.
        Thus, as $-1 \leq 0 \leq a+1$, by the intermediate value
        theorem, there exists some $c \in [0, 1]$ such that $f(c) = 0$.

        That is, there is \textit{a} solution for $ax = \cos \pi x$ on
        the specified interval.

        Further, on the interval $[0, 1]$,
        $$f'(x) = a + \pi \sin \pi x > 0$$
        that is, $f$ is increasing, and so it can have at most one root
        in that region.

        It follows that $ax = \cos \pi x$ has exactly one solution on
        the interval $[0, 1]$, thus completing the proof.
    \end{proof}
\end{enumerate}
