\centering{\bf \Large 2009 Version 1a}

\begin{enumerate}
% Question 1, logic
\item ~

    \begin{tabular}{l l}
    $(p \rightarrow q) \wedge (q \rightarrow (\sim p \vee r))$ &  \\
    $\Leftrightarrow ((\sim p) \vee q) \wedge ((\sim q) \vee (\sim p \vee r))$
        & as $v \rightarrow u \Leftrightarrow (\sim v) \vee u$ \\
    $\Leftrightarrow ((\sim p) \vee q) \wedge ((\sim p) \vee (\sim q \vee r))$
        & Associative, commutative laws \\
    $\Leftrightarrow (\sim p) \vee (q\, \wedge (\sim q \vee r))$
        & Distributive law \\
    $\Leftrightarrow (\sim p) \vee ((q\, \wedge \sim q) \vee (q \wedge r))$
        & Distributive law \\
    $\Leftrightarrow (\sim p) \vee (\mathbb{F} \vee (q \wedge r))$
        & Law of negation \\
    $\Leftrightarrow (\sim p) \vee (q \wedge r)$
        & Identity law \\
    $\Leftrightarrow p \rightarrow (q \wedge r)$
        & as $v \rightarrow u \Leftrightarrow (\sim v) \vee u$ \\
    \end{tabular}

    as required.

% Question 2: equality proof
\item
    \begin{theorem}
        If $n \in \Z^+$, tehn
        $$(n+1)(n+2)\dots(2n) = 2^n \times 1 \times 3 \times \dots \times (2n-1)$$
    \end{theorem}
    \begin{proof}
        Let $n \in \Z^+$.

        Now,
        \begin{align*}
            \textrm{LHS} &= (n+1)(n+2)\dots(2n) \\
            &= \frac{1 \times 2 \times 3 \times \dots \times n \times (n+1) \times \dots \times (2n)}{1 \times 2 \times 3 \times \dots \times n} \\
            &= \frac{(2n)!}{n!}
        \end{align*}

        And,
        \begin{align*}
            \textrm{RHS} &= 2^n \times 1 \times 3 \times \dots \times (2n-1) \\
            &= 2^n \times \left( \frac{1 \times 2 \times 3 \times \dots \times 2n}{2 \times 4 \times 6 \times \dots \times 2n} \right) \\
            % kill me now
            &= (2n)! \times\frac
            {\overbrace{2 \times 2 \times \dots \times 2}^{n \textrm{ terms}}}
            {\underbrace{2 \times 4 \times 6 \times \dots \times 2n}_{n \textrm{ terms}}} \\
            &= (2n)! \times \frac{1}{1 \times 2 \times 3 \times \dots \times n} \\
            &= \frac{(2n)!}{n!} \\
            &= \textrm{LHS}
        \end{align*}

        The result follows, thus completing the proof.
    \end{proof}

% Question 3: irrationality
\item
    \begin{theorem}
        If $n \in \Z^+$, then $\sqrt{4n-2}$ is irrational.
    \end{theorem}
    \begin{proof}
        We prove this by contradiction.

        Let $n \in \Z^+$.
        Suppose that $\sqrt{4n-2}$ is rational, that is, it can be
        written as $\frac{p}{q}$, with $p, q \in \Z$.

        Without loss of generality, we assume that $p$ and $q$ share no
        common factors.

        Then,
        \begin{align*}
            \sqrt{4n-2} &= \frac{p}{q} \\
            4n-2 &= \frac{p^2}{q^2} \\
            2(2n-1) &= \frac{p^2}{q^2} \\
            \Rightarrow 2(2n-1)q^2 &= p^2
        \end{align*}

        (Since the above are integers) It follows that $p^2$ is even,
        and so $p$ is also even.
        Suppose $p = 2m$, for some $m \in \Z$.

        Then we have,
        \begin{align*}
            2(2n-1)q^2 &= p^2 \\
            2(2n-1)q^2 &= 4m^2 \\
            (2n-1)q^2 &= 2m^2 \\
        \end{align*}

        Similarly, we see that $q$ is also even.
        But since $p$ and $q$ share no common factors, this is a
        contradiction.

        Thus, the initial assumption was wrong, and it follows that
        $\sqrt{4n-2}$ is irrational for all $n \in \Z^+$, completing
        the proof.
    \end{proof}
\end{enumerate}
