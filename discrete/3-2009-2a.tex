\centering{\bf \Large 2009 Version 2a}

\begin{enumerate}
% Question 1, logical equivalences
\item ~

    \begin{tabular}{ll}
        $p \rightarrow (\sim (q \wedge (\sim p)))$ & \\
        $\Leftrightarrow(\sim p) \vee (\sim (q \wedge (\sim p)))$ &
            as $v \rightarrow u \Leftrightarrow (\sim v) \vee u$ \\
        $\Leftrightarrow\sim(p \wedge (q \wedge (\sim p)))$ &
            De Morgan's law \\
        $\Leftrightarrow\sim(p \wedge (\sim p) \wedge q)$ &
            Commutative and associative laws \\
        $\Leftrightarrow\sim(\mathbb{F} \wedge q)$ & Law of negation \\
        $\Leftrightarrow\sim(\mathbb{F})$ & Domination law \\
        $\Leftrightarrow\mathbb{T}$ & Negation of a contradiction \\
    \end{tabular}

% Question 2: induction
\item
    \begin{theorem}
        If $n \in \Z^+$, then
        $$(1 \times 2) + (2 \times 5) + \dots + n(3n-1) = n^2(n+1)$$
    \end{theorem}
    \begin{proof}
        We prove by induction.

        When $n = 1$, LHS $= 1 \times 2 = 2$, RHS $= 1^2 \times 2 = 2$,
        and so the result is true for $n=1$.

        Assume the result for $n=k$, where $k \in \Z^+$. That is:
        $$(1 \times 2) + (2 \times 5) + \dots + k(3k-1) = k^2(k+1)$$

        Consider $n = k+1$. We have:
        \begin{align*}
            \textrm{LHS} &= (1 \times 2) + \dots + k(3k-1) + (k+1)(3k+2) \\
            &= k^2(k+1) + (k+1)(3k+2) & \textrm{(by assumption)} \\
            &= (k+1)(k^2 + 3k + 2) \\
            &= (k+1)(k+1)(k+2) & \textrm{(factorising)} \\
            &= (k+1)^2((k+1)+1) \\
            &= \textrm{RHS}
        \end{align*}

        Hence, if the result is true for $n=k$ for some $k \in \Z^+$,
        then it is also true for $n=k+1$.

        Therefore, the result is proved for all positive integers $n$,
        by induction, and the proof is complete.
    \end{proof}

% Question 3: irrationality
\item
    \begin{theorem}
        If $x \in \R$ and $2x^2 - 3 = 0$, then $x$ is irrational.
    \end{theorem}
    \begin{proof}
        We prove by contradiction.

        Let $x \in \R$ and $2x^2 - 3 = 0$.
        Solving, we obtain
        $$x^2 = \frac{3}{2}$$

        Suppose that $x$ is rational, that is, it can be written as an
        irreducible ratio of two integers, $x = \frac{p}{q}$ for some
        $p, q \in \Z \backslash \{0\}$.

        Then, $x^2 = \frac{p^2}{q^2} = \frac{3}{2}$ and thus
        $$2p^2 = 3q^2$$

        Clearly, the LHS is even, and so $q^2$ must be even also.
        This implies that $q$ is even, and hence $q = 2r$ for some
        $r \in \Z$.

        Simplifying,
        $$p^2 = 2 \times 3 \times r^2$$

        By similar argument, we see that $p$ must be even, and hence
        $p$ and $q$ share a common factor of 2.
        But since we assumed that $p$ and $q$ form an irreducible
        fraction, this is a contradiction.

        Thus, the initial assumption was wrong.

        It follows that $x$ is irrational, and the proof is complete.
    \end{proof}
\end{enumerate}
