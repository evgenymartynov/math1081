\centering{\bf \Large 2008 Version 3a}

\begin{enumerate}
% Question 1, truth tables
\item
    \begin{enumerate}
    \item 
        $\sim p \rightarrow (q \wedge p)$

        \begin{tabular}{c|c|c|c|c}
        $p$ & $q$ & $\sim p$ & $q \wedge p$ & $\sim p \rightarrow (q \wedge p)$ \\
        \hline
        T & T & F & T & T \\
        T & F & F & F & T \\
        F & T & T & F & F \\
        F & F & T & F & F \\
        \end{tabular}

        \vspace{3mm}

        $(p \wedge q) \rightarrow q$

        \begin{tabular}{c|c|c|c}
        $p$ & $q$ & $p \wedge q$ & $(p \wedge q) \rightarrow q$ \\
        \hline
        T & T & T & T \\
        T & F & F & T \\
        F & T & F & T \\
        F & F & F & T \\
        \end{tabular}

    \item
        First off, note that the second formula cannot imply the first
        when $p = q = F$.

        On the other hand, the first formula \textbf{does} imply the
        second, since when the first is true, the second is true also.

        This can be better explained using this truth table:

        \begin{tabular}{c|c|c|c|c}
        $p$ & $q$ & $\sim p \rightarrow (q \wedge p)$ &
            $(p \wedge q) \rightarrow q$ & first $\rightarrow$ second \\
        \hline
        T & T & T & T & T \\
        T & F & T & T & T \\
        F & T & F & T & T \\
        F & F & F & T & T \\
        \end{tabular}

        Hence, the first formula logically implies the second.

    \end{enumerate}

% Question 2, log_6 (11) not in Q
\item
    \begin{theorem}
        $\log_6 11$ is irrational.
    \end{theorem}
    \begin{proof}
        We prove by contradiction.

        Suppose that $\log_6 11$ is rational, that is, there exist
        integers $p$ and $q$ such that $log_6 11 = \frac{p}{q}$.
        Without loss of generality, we can assume $p, q > 0$.

        Then,
        \begin{align*}
            \log_6 11 &= \frac{p}{q} \\
            11 &= 6^\frac{p}{q} \\
            11^q &= 6^p \\
        \end{align*}

        which is a contradiction, as the LHS is always odd and RHS is
        always even.

        Hence, our assumption was false, and it follows that $\log_6 11$
        is irrational, thus completing the proof.
    \end{proof}

% Question 3, the one with prime numbers.
\item
    \begin{theorem}
        $q(n) = 11n^2 + 32n$ is a prime number for two integer values
        of $n$, and composite for every other integer values of $n$.
    \end{theorem}
    \begin{proof}
        If $q(n)$ is prime, then it has exactly two factors, which are
        1 and $q(n)$.

        Factorising $q(n)$, we obtain $q(n) = n(11n+32)$.
        From the definition, it is necessary that at least one of
        $n = \pm 1$ or $11n + 32 = \pm 1$ are true.

        We now consider the possible cases.

        If $n = 1$,  then $q(n) = 43$, which is prime.

        If $n = -1$, then $q(n) = -21$, which is not prime.

        If $11n + 32 = 1$, then $n = \frac{-31}{11} \not \in \Z$.
        This contradicts the constraint of $n \in \Z$ and hence is not
        possible.

        If $11n + 32 = -1$, then $n = -3$, and hence $q(n) = 3$,
        which is prime.

        By exhaustion of possible cases where $q(n)$ can be prime, we
        conclude that $q(n)$ is prime in only two cases (namely, $n=1$
        or $n=-3$), and composite for every other integer value of $n$.
    \end{proof}
\end{enumerate}
